\documentclass[12pt]{article}
\usepackage[a4paper, left = 35mm, top = 24mm, bottom = 24mm, right = 24mm]{geometry}
\usepackage{amsfonts, amsmath, amssymb}
\usepackage{textcomp}
\usepackage{gensymb}
\usepackage{fancyhdr}
\usepackage{pdfpages}
\usepackage{listings}
\usepackage{color}
\usepackage{pdfpages}
\usepackage{hyperref}
\usepackage{verbatim}
\usepackage[utf8]{inputenc}
\usepackage{graphicx}
\usepackage{float}
\usepackage{longtable}
\usepackage{fontenc}
\usepackage{times}
\usepackage[ddmmyyyy]{datetime}
\usepackage{pdfpages}
\usepackage{subcaption}
\usepackage{multirow}
\usepackage{epstopdf}
\usepackage{dirtree}
\usepackage{listings}

%To make captions centered and italic
\usepackage[format=plain, labelfont=bf, textfont=it, justification=centering]{caption}
%To add custom indents to bullet points
\usepackage{enumitem}


\pagestyle{empty}

\setlength{\parindent}{0pt}
\setlength{\parskip}{1em}

\begin{document}

%//////////////////////////////////////////////////////////////////////////////////////////////
%///////////////////////////////////// Title Page /////////////////////////////////////////////
%//////////////////////////////////////////////////////////////////////////////////////////////


\begin{titlepage}

    \newgeometry{left=4cm, right=4cm, top=4cm, bottom=4cm}
    \begin{center}
        
        {\LARGE The Electronic Hurdy Gurdy Project}

        {\Large Pete Alexander}
        \vspace{5cm}
        
        {\LARGE Project Document}

    \end{center}
    \restoregeometry
\end{titlepage}

%//////////////////////////////////////////////////////////////////////////////////////////////
%/////////////////////////////////// End Title Page ///////////////////////////////////////////
%//////////////////////////////////////////////////////////////////////////////////////////////


\newpage
\setcounter{tocdepth}{2}

%TC:ignore
\newpage
\tableofcontents
\newpage

\setcounter{page}{1}
\pagestyle{plain}

\section{Introduction}
\subsection{Document Overview}
This is the project document for my Hurdy Gurdy project. 
It serves as the central log of the aims, direction and scope of this project.
Any change to any of these things will be accompanied by a release of this document. 
I'm developing this project in a pretty agile way so this won't have too much detail (plus I'm doing this for fun).

\subsection{Project Overview}
This project aims to build an Electronic Hurdy Gurdy (google it).
In its first iteration will be a MIDI instrument capable of emulating the interfaces of a traditional Hurdy Gurdy. 
It is to be constructed to the same standard I hold my other musical instruments to (i.e. production-level).
The following is a non-exhaustive list of the project's major parts.

\begin{enumerate}
    \item PCBs for control electronics and key interface.
    \item Switch/Button interface for engaging the different virtual strings of the instrument. 
    \item Firmware for control CPU to convert interface inputs to USB MIDI commands. 
    \item 3D Printed Chassis and extras to hold electronics and interfaces.
    \item Crank handle with motor interface. 
\end{enumerate}

The following is a list of extensions to the project that will follow the base prototype:

\begin{enumerate}
    \item Screen and encoder for selecting properties of the instrument (e.g. drone/trompet key).
    \item "Button Board" add-on to crank handle for changing drone notes accordion-style.
    \item Synthesizer add-on for on-board sound generation. 
\end{enumerate}


\section{Phases}
This project will be split into phases to group key decision points by type.

\subsection{Phase 1: Investigation}
To start, different implementations of key features will be investigated to choose the best one. 
This can come in the form of practical work (e.g. trying different MCU development boards).
Or it can be paper work (e.g. finding a source of cheap aluminium extrusions). 

\subsection{Phase 2: Development}
Once all decision have been made, I will begin the development of the real hardware.
This will be all of the non-destructive work that goes into the first prototype.
This will be things like PCB design, code writing on dev-boards, testing small 3D printed parts.  

\subsection{Phase 3: Prototype Construction}
Once I've got all the designs for parts finished to a good standard, I will construct the first prototype.
This will be a playable instrument that should produce "tight" music. 
It will be pretty feature light but the design should enable easy expansion/tinkering. 

\subsection{Phase \(\infty\): Expansion and Maintenance}
Once the prototype is built, it's time to build on that success!
I intend to use one version or another of this thing indefinitely. 
This will require maintenance, probably quite a lot at first.
I'd also like to expand features on top of the basics.
I'd like to add some extra features and to tune it to my way of playing. 

\end{document}